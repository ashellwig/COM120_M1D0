%! TEX root='..\main.tex'

\section{Response Posts}
  \subsection{Response 1}
    \begin{quotation}
      Hello everyone, my name is Sabrina, I live in Denver Co. My major is Early
        Childhood Education, I am currently attending Community College of
        Aurora, my career goal is to finish my major and hopefully get a good
        job in my field. The reason that I enrolled in an online class because
        it is required for my major, also it seems like an interesting subject
        to learn about. My hobbies are cooking, photography, and knitting, my
        favorite travel destination is Italy and California Beaches, my
        favorite quote has to be by Alan Watts “This is the real secret of life
        --- to be completely engaged with what you are doing in the here and
        now. And instead of calling it work, realize it is play.”. My goal to
        complete and submit classwork in accordance with due dates would be to
        plan ahead of time and look at the syllables to check and see what is
        due and what I am supposed to work on.

      There was a time where my friend was texting me constantly and I could
        only give them a short reply because I was very busy and couldn’t really
        focus on what was going on and what they were saying in the text message
        which made my friend upset and thought that I don’t want to talk to
        them, and they thought that I am mad at them, although they did ask me
        if I was okay and asked if I am mad at them which was really upsetting
        because it wasn’t my intention to get my friend sad and think that I am
        mad them for no reason, however, I did let them know that I was busy and
        the reason that I did not pay much attention was that I was busy and
        finishing up my homework.

      The problem lied with their understanding and interpretation of the
        message because we were texting and I couldn’t really show that much
        expression it made my friend upset. It is quite hard to express your
        emotions and sometimes people would have a completely different
        understanding of your message and that’s mainly because a lot of times
        we are texting and we don’t really show emotion and can’t really do so
        when we are texting so that could be a cause in my opinion. I guess I
        could have tried to pay more attention to their messages but sometimes
        when you are very busy it would be hard to focus on everything at the
        same time.
    \end{quotation}

    \paragraph{This is a response to Sabrina Barzinjy with Post ID 43158270}
      Hello, Sabrina! It is great to see more people in Early Childhood
        Education, many of my friends are in that field. I wonder, how do you
        adjust how you converse with others based on their age? I imagine how
        you would speak with your peers here in class versus friends outside of
        school is completely different than how you would speak with one of
        your student's teachers or with the student him, her, or however they
        may identify. Is it difficult to have so many different expectations
        from those you would be speaking with everyday and maintaining the
        correct mode-of-speech for each individual conversation?
