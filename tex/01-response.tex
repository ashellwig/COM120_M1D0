%! TEX root='..\main.tex'

\section{Response Posts}
  \subsection{Response 1}
    \begin{quotation}
      In my opinion, the five traits that are extremely crucial in being an
        effective interpersonal communicator include active listening,presenting
        good body language, utilizing problem-solving skills, displaying empathy
        and maintaining a positive attitude. The two skills that I struggle with
        the most are displaying empathy and being able to maintain a positive
        outlook. As a registered behavioral technician, a fundamental skill that
        I have learned to acquire is the ability to listen, observe and be more
        emotionally aware of my client’s struggles, in order to better deliver
        individualized, behavioral therapy.

      One client that I work with is a 4-year-old girl, G.G., who is diagnosed
        with Autism Spectrum Disorder, as well as Rett syndrome. G.G. is
        nonverbal and will most likely never learn how to speak or independently
        take care of herself.  These harsh facts weigh heavy on my heart and
        therefore I feel responsible for teaching her skills that can
        potentially improve her quality of life.  I work several times a week
        with G.G. teaching her alternative ways to communicate her wants and
        needs, by incorporating pictures of items she desires or through a
        communication device. G.G. still has a long road ahead before she is
        capable of properly utilizing an alternative communication method.
        She has good days, when it appears she is utilizing the communication
        methods I have taught her and consequently will be in good spirits. On
        these days, I praise her and feel overjoyed by her accomplishment in
        overcoming one of her greatest obstacles.  I know that I am making
        progress with her because she makes eye contact with me. She
        acknowledges my verbal cues and responds to me with joy. Contrastingly,
        she also experiences bad days, when she lashes out in frustration
        because she is incapable of communicating what she needs in that moment.
        On G.G.’s worst days, witnessing her struggle weighs heavy on me and it
        is difficult for me to maintain a positive attitude when I am trying to
        engage with her.  I have to be able to take all of the visual and
        non-verbal information from our interactions to understand how to
        respond to her. At times, she can read my negative, closed off body
        language and as a result, GG becomes more anxious.

      Communication is a vital skill that is often taken for granted. The key to
        good communication is to first listen and seek to understand. I can
        never truly empathize with how G.G. feels not being able to speak nor to
        be heard.  However, by attempting to gain understanding of G.G.’s world,
        I have begun to help her confront her communication limitations. This
        skill in turn allows me to better understand her and to be a better
        educator.
    \end{quotation}

    \paragraph{This is a response to Daniella Maduro with Post ID 43200009}
    I feel it it is very important to have someone's perspective on
      interpersonal communication as behavioral therapy technician. I also can
      understand the difficulty in presenting a comforting appearance when
      attempting to communicate with \textit{anybody} having one of their
      ``bad days''. When you had mentioned that during GG's bad days, she can
      sometimes become more anxious if she can sense your own anxiety or fear
      when face without knowledge of how to handle a certain situation I was
      wondering if you happened to have yet found a good way to remain in a
      ``good light'', so to speak, during difficult interactions.
